\begin{abstract}

現在,産業用ロボットは多種多様な作業を遂行するためにエンドエフェクタの交換をおこなっている.
把持や運搬,組み付け作業を行うときのものをグリッパと呼ばれ,対象物の姿勢や形状に合わせて適切なグリッパへの交換が一般的に行われている.
しかし,グリッパの交換には複雑な把持計画や交換作業が伴い効率的な作業の障害となる問題が存在する.
また自動車部品の製造工場などでは上記の問題に加えて車種固有の意匠部品などは形状の多岐に及んでいるため把持の複雑さや部品自体がデリケートなものによることから従業員の手作業で組み付けが行われており,自動化できていない課題がある.\par
こうした問題を解決するために近年把持対象物の姿勢認識とグリッパの交換を省略し,作業効率を向上させるユニバーサルグリッパと呼ばれるものの開発が行われている.
ユニバーサルグリッパの中でも把持部に柔軟性のあるグリッパは,把持対象物を包み込むことで把持部を対象物に密着させて接触面積を増やし,対象物との間に生じる摩擦を増やして柔軟な把持を可能にする.
また,はや戻り機構を有するグリッパは高速で把持と開放を高速で行うことができタクトタイムの減少を期待できる.\par
本研究では,柔軟な把持を可能にする指をもちはや戻り機能を有するグリッパの提案をする.


\end{abstract}
\thispagestyle{empty}

\begin{abstract}
現在,日本の人口減少における工場の労働者不足を補うためにロボットによる省人化が求められている.その中で産業用ロボットは様々な作業に幅広く対応するためにエンドエフェクタの交換を行う.把持,搬送,組付け作業を行うエンドエフェクタはグリッパと呼ばれる.グリッパの交換には作業ごとのグリッパ選定や把持計画の再実施を行わなければならず作業効率の低下につながる.この問題を解決するため汎用的に作業できるユニバーサルグリッパの開発が行われている.昨今ユニバーサルグリッパには汎用的な把持性能だけでなく組付け作業に対応可能な力覚機能の需要が増えている.力覚機能を有するロボット用の指研究の多くは産業用途対応する耐久性を持たない.そこで本研究では耐久性と把持性能を両立する柔軟指に力覚機能を付加させ性能を検証した.ラティス構造を有する2種類の柔軟指を作成し汎用的な把持のできる内骨格型グリッパを用いた.
\end{abstract}
\section{実験}
\subsection{基礎実験}
作成した柔軟指に力覚センサ取り付けたものに荷重実験を行った.ここでは柔軟指に\refig{force_gauge}に示すフォースゲージを用いて一定の荷重を加えセンサの値を計測する.計測誤差結果を\refig{result0}に示す.

\subsubsection{実験手順}
\begin{enumerate}
  \item 柔軟指に力覚センサを仕込み.  
  \item グリッパの指を開き作業台の鉛直上方向から把持に適切な位置まで接近させた. 
  \item グリッパの指を閉じ把持対象物を把持した.
  \item 把持対象物を把持したまま鉛直上方向に持ち上げることができれば把持可能,持ち上げることができなければ把持不可能と判別した.
\end{enumerate}

\subsection{結果と考察}
荷重実験の結果より期待通りの応答を得ることはできなかった.これは柔軟部のラティス構造が力を分散しセンサ部に適切に力が加わらなかったからであると考えられる.




%\subsection{把持実験}
%本グリッパの指を入れ替えて対象物の把持の可不可を検証する.指の種類はゲル,バネ,ゴムで被覆したスポンジ,柔軟物なしのものとし以下にまとめる.検証に用いる把持対象物を\refig{denso_parts}に示す.把持対象物は自動車部品工場で用いられる部品のモデルの5種類とし,それぞれA,B,C,D,Eの記号を部品ごとに割り振った.
%また,把持対象物Eは\refig{E}上部の突起部と下部の外周部でそれぞれ把持実験を行った.






\newpage


%\subsubsection{実験結果}
%使用した指は以下の通りである.実験結果を\reftab{result}にまとめる.把持成功は◯,把持失敗は×を表している.把持対象物Eは上部の突起部と下部の外周部での把持が両方成功した時把持成功とみなした.\reftab{result}の横軸は\refig{denso_parts}の割り振った記号と対応し縦軸は以下の指の番号と対応している.

%\begin{enumerate}
  %\item 柔軟物のない指
  %\item 厚さ3mmのゲルを取り付けた指
  
  
%\end{enumerate}

%\begin{table}[htbp]
 %   \caption{把持実験結果}
 
%  \label{tab::result}
   %\scalebox{3}[1.5]
  % \centering
   %\begin{tabular}{|c||c|c|c|c|c|} \hline
      %    &A    &B     &C      &D     &E        %\\ \hline \hline
 %       (1) & ◯ & ◯  & ×  & ◯ & ×  \\ \hline
   %     (2) & ◯ & ◯  & ◯  & ◯ & ◯  \\ \hline
    %    (3) & ◯ & ◯  & ◯  & ◯ & ◯  \\ \hline
	%	(4) & ◯ & ◯  & ◯  & ◯ & ×  \\ \hline
	%	(5) & ◯ & ◯  & ◯  & ◯ & ×  \\ \hline				
	%	(6) & ◯ & ◯  & ◯  & ◯ & ×  \\ \hline
	%	(7) & ◯ & ◯  & ◯  & ◯ & ×  \\ \hline
	%	(8) & × & ×  & ×  & × & ×  \\ \hline		
	%	(9) & × & ×  & ×  & × & ×  \\ \hline		
			
		
    %\end{tabular}
%\end{table}

%\begin{figure}[h]
%\centering
%\subfloat[把持対象物A]{\includegraphics[scale=0.4]{../figure/result_a.eps}}
%\hspace{5mm}
%\subfloat[把持対象物B]{\includegraphics[scale=0.4]{../figure/result_b.eps}}
%\hspace{5mm}
%\subfloat[把持対象物C]{\includegraphics[scale=0.4]{../figure/result_c.eps}}
%\hspace{5mm}
%\subfloat[把持対象物D]{\includegraphics[scale=0.4]{../figure/result_d.eps}}
%\hspace{5mm}
%\subfloat[把持対象物E 上部の突起]{\includegraphics[scale=0.4]{../figure/result_e1.eps}}
%\hspace{5mm}
%\subfloat[把持対象物E 下部の外周部]{\includegraphics[scale=0.4]{../figure/result_e2.eps}}
%\hspace{5mm}
%\caption{厚さ3mmのゲルの指での把持の様子}
%\label{fig::sisaku}
%\end{figure}



\section{序論}
\label{sec:序論}
産業用ロボットは一般的に把持、搬送などの作業をグリッパと呼ばれるエンドエフェクタを用いる.グリッパは各作業用に専用設計されていることが多く作業工程が移り変わるたびに交換がなされており、最適なグリッパの選定や各グリッパごとに複雑な把持計画を必要とする.近年グリッパの交換をせずとも様々な作業を遂行できる汎用性の高いグリッパの研究がなされている.汎用グリッパにも様々な種類がありそれらは多関節グリッパ、柔軟グリッパ、内骨格型柔軟グリッパに大別される.多関節グリッパは複数の関節をもつ指を持ちそれらが対象物に倣うことで把持を行う\cite{takansetsu}.多関節グリッパの代表例としてROBOTIQ社のROBOTIQ ADAPTIVE GRIPPER 3-FINGER MODELを\refig{robotiq}に示す.問題点は把持時の接触部を増やすために指の関節を増やすことが求められ機構が複雑になる点が挙げられる.
柔軟グリッパは把持部に柔軟膜を有し把持対象物の形状にあわせて変形する特徴がある.対象物に倣い接触面積と摩擦力と増やすことで把持を行う.柔軟グリッパの例として以下の2つがある.ジャミンググリッパは柔軟膜の中に空気と粒体を有しジャミング転移現象を利用して固化し把持を行う\ref{jamming}.
また半球状の柔軟膜の中にMR流体を封入したMR流体グリッパ\cite{MR}がある.
このグリッパはMR流体に磁界を印加することで固化する特徴がある.
内骨格型柔軟グリッパは内骨格と言われる固い指の表面に柔軟膜を有している.柔軟指部分が把持対象物の形状に倣いつつ内骨格による力拘束が可能である.以上の特徴から対象物の位置決め誤差や計測誤差にロバストであるとともに比較的重い物体の把持が可能である.


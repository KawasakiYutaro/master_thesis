\section{序論}
\label{sec:序論}
産業用ロボットは一般的に把持、搬送などの作業をグリッパと呼ばれるエンドエフェクタを用いる.グリッパは各作業用に専用設計されていることが多く作業工程が移り変わるたびに交換がなされており、最適なグリッパの選定や各グリッパごとに複雑な把持計画を必要とする.近年グリッパの交換をせずとも様々な作業を遂行できる汎用性の高いグリッパの研究がなされている.汎用グリッパにも様々な種類がありそれらは多関節グリッパ、柔軟グリッパ、内骨格型柔軟グリッパに大別される.多関節グリッパは複数の関節をもつ指を持ちそれらが対象物に倣うことで把持を行う\cite{takansetsu}.多関節グリッパの代表例としてROBOTIQ社のROBOTIQ ADAPTIVE GRIPPER 3-FINGER MODELを\refig{robotiq}に示す.問題点は把持時の接触部を増やすために指の関節を増やすことが求められ機構が複雑になる点が挙げられる.
柔軟グリッパは柔軟膜と内部に封入された流体で構成さっふぇれる.把持対象物を柔らかく包みこんだ後ジャミング転移現象を利用して固化することで把持を可能にする.












半球状の柔軟膜の中にMR流体を封入したMR流体グリッパ\cite{MR}がある.
このグリッパはMR流体に磁界を印加することで粘性が変化することを応用し把持に用いる.これらのような柔軟な把持は把持対象物の形状,姿勢に左右にされない強みがある.\par
またグリッパの把持動作の高速化はタクトタイムの短縮になり,産業用ロボットの作業効率化,生産性向上につながると考えられる.指の開閉の高速化により把持動作の高速化を可能としたグリッパに早戻り機構を用いたグリッパがある\cite{sh_hand}.\par
本研究では早戻り機構を用いたグリッパに焦点を当てた.早戻り機構を有するグリッパの把持部に柔軟性をもたせることで高速な開閉が可能なグリッパに把持対象物の形状や姿勢によらない汎用性を付加可能か検証する.

